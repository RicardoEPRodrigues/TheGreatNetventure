\chapter{Protótipos}
\label{chap:proto}
%A description of all (functional and non-functional) prototypes developed together with an explanation of their development path (e.g., a discussion of any alternatives that were considered).
A construção dos protótipos começou após uma reunião de grupo onde se acordou que cada elemento criaria 2 (dois) protótipos não-funcionais para um jogo que eles achassem que seria um bom jogo. Estes foram os resultados.
\newline
\textbf{Protótipos do Ricardo}


\newline
\textbf{Protótipos do Tiago}


\newline
\textbf{Protótipos do Ian}
A primeira ideia foi um rail-shooter, onde se controlava uma nave com uma mensagem e disparava caractéres de password. O objectivo seria levar a mensagem de um ponto a outro, protegendo de vários vírus e bosses que iriam aparecendo ao longo dos níveis. Seria também possível apanhar power ups, como escudos de Anti-Virus. Haveria dois modos, um modo campanha que seria uma série de níveis, e um modo infinito, que seria um nível que apenas acabaria quando o jogador ficasse sem vidas. Este modo infinito permitiria guardar a pontuação numa tabela para comparar com a de outros jogadores.

A outra ideia era uma espécie de God Game, onde o jogador controlava um Deus das Nuvens (Clouds). Esse Deus seria venerado inicialmente por uma vila, e o objectivo seria expandir essa vila, ganhando assim pontos. Com estes pontos seria possível desbloquear poderes, como por exemplo Firewalls para proteger a vila de inimigos. Os inimigos poderiam ser vários tipo de vírus (Worms seriam representados por serpentes gigantes, Trojans cavalos, etc), ou outras vilas hostis, representando Hackers. A movimentação dos habitantes pelas vilas seriam assim uma metáfora para a troca do informação pela Cloud, e a expansão da vila o crescimento da Cloud.
\newline