\chapter{Cenários de Utilização}
\label{chap:usage} 
%Complete descriptions of all final usage scenarios

Numa primeira fase começámos por definir apenas um cenário de utilização para os alunos e outro para a professora, que seria uma versão optimista dos mesmos. Posteriormente decidimos acrescentar um outro tipo de cenário, que seria um mais pessimista em que tudo corria mal e que permitia verificar quais os perigos ou problemas na utilização de alguns serviços de Cloud.

\begin{itemize}

\item \textit{\underline{Professor - Cenário Um}}\\

A Professora Ana tinha uma aula para preparar. Para tal, decidiu usar \textbf{uma apresentação em slides} que tinha feito com um colega no ano lectivo anterior, infelizmente, por não usar um sistema de Cloud de partilha de ficheiros, não sabia onde tinha o ficheiro.

Para resolver esse seu problema decidiu ver as suas \textbf{conversas com o seu colega no Facebook}, encontrou alguns ficheiros relacionados com o tema, mas não o que procurava. Assim não teve outra hipótese senão \textbf{enviar um email} ao seu colega pedindo o ficheiro.

Enquanto esperava decidiu jogar o seu \textbf{jogo de simulação preferido, Farmville} (ia jogar de qualquer das maneiras, os cereais não se colhem sozinhos!). Finalmente recebeu um email de retorno! O seu colega enviara-lhe a apresentação em anexo. É melhor desta vez não perder! Ufa!

\item \textit{\underline{Professor - Cenário Dois}}\\

A professora Ana preparou uma \textbf{apresentação em Powerpoint e uns vídeos} para a sua aula no dia seguinte e antes de desligar o computador decidiu alterar um gráfico no documento. Após terminar decidiu ir dormir e deixar o computador em cima da mesa.

No dia seguinte acordou e reparou que já devia estar acordada há pelo menos meia hora, despachou-se e foi a correr para o carro.

Entretanto a aula começa e quando vai à mala retirar o computador, repara que ele não está lá. Oh não e agora?? Entretanto lembrou-se que guardou a apresentação na \textbf{Dropbox} e que havia computadores portáteis na escola. Pediu um à funcionária e quando \textbf{abriu o navegador de Internet} reparou que a ligação não estava disponível. Primeiro o computador, agora a Internet não funciona? 

Como a aula desse dia era apenas mostrar alguns gráficos, não levou quaisquer apontamentos para a aula e após trinta minutos decidiu terminar a aula. Será que esta dia pode correr pior?

\item \textit{\underline{Alunos - Cenário Um}}\\

AH! Quinta-feira! O dia perfeito para a Mónica passar a tarde a \textbf{jogar um dos seus jogos favoritos, CrossFire}. Os dias mais estudiosos da semana já passaram e à sexta-feira só se tem duas aulas.

Quatro horas passaram e a Mónica ganhava todos os jogos que fazia! Decidiu fazer uma pausa, desligou o jogo, e \textbf{visitou o seu Facebook}. Assustou-se! O seu colega Rui não lhe parava de enviar mensagens, “Vai ao teu E-Mail!”, “Rápido tens de ler aquilo!”. Em pânico, ela abriu a \textbf{sua conta de correio electrónico} e foi então que viu um e-mail do seu professor de Português.

“Amanhã, sexta-feira, deverá ser feita uma apresentação por grupos sobre como Camões perdeu o olho, a apresentação deve…”

Ia desmaiando. Pensava estar descansada quando o maldito professor a tramou! Respirou fundo e acalmou-se, a apresentação não há de ser assim tão difícil. Voltou ao Facebook, afinal se não fosse o seu parceiro de grupo ela ainda estaria no escuro. Após uma conversa com o Rui, combinou-se como o trabalho seria feito e que o seu colega já tinha criado uma \textbf{pequena apresentação}.

“Podes-me mandar a apresentação?” - Perguntou a Mónica.
Rui explicou-lhe que usa o Google Drive, um \textbf{sistema de partilha de ficheiros que usa a cloud}, e como ela o podia usar. Depois da apresentação feita e de tudo combinado para a sua aula, a Mónica respirou fundo e foi-se deitar, afinal de contas já era uma hora da manhã.

\item \textit{\underline{Alunos - Cenário Dois}}\\

O Rúben tinha fotos suas de bebé a tomar banho guardadas pela mãe num \textbf{sistema de partilha de ficheiros, o Meo Cloud}. Para além das fotos, os \textbf{dados de login do Facebook} dele estavam lá guardados. 

Certo dia um \textbf{hacker} conseguiu aceder aos dados de algumas das contas deste serviço e uma delas foi a conta da mãe do Rúben. O hacker acedeu ao Facebook do Rúben, alterou a password e no perfil encontrou o número de telemóvel dele. No dia seguinte o telemóvel do Rúben toca, era uma mensagem a dizer se queria a conta do Facebook de volta e as fotos, teria que pagar 50 euros. Chantagem! Tentou entrar na conta, mas sem sucesso... Com receio do que o hacker podia fazer com a conta dele e com as informações que lá estavam sobre ele e os seus familiares, decidiu juntar todas as suas poupanças e enviar o dinheiro. Será que ele vai devolver a conta?
\end{itemize}
