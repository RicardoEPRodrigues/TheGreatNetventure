\chapter{Resultados dos Testes de Usabilidade}
\label{chap:results} 
%The results obtained from applying usability tests to all iterations/prototypes (showing the progress across iterations), including the final tests.
\section{Protótipos de Baixa Fidelidade}
Os primeiros testes de usabilidade realizados foram aos protótipos de baixa fidelidade e consistiram em avaliar o funcionamento dos menus e o movimento da nave. Como tal obtivemos os seguintes resultados para os testes.\\

\textbf{Teste 1 - Tempo para aceder a informação sobre \textit{spywares}}

\begin{itemize}

\item Resultados: 10 segundos.

\item Resultados mínimos atingidos.

\item Indica a necessidade de alteração da área de informação. 
\end{itemize}

\textbf{Teste 2 - Tempo para escolher o nível 1}

\begin{itemize}

\item Resultados: 5 segundos.

\item Teste cumprido com sucesso.

\item É possível que os valores escolhidos na medida sejam altos.
\end{itemize}

\textbf{Teste 3 - Erros cometidos para seleccionar o nível 1}

\begin{itemize}

\item Resultados: 0 erros.

\item Resultados óptimos alcançados.

\item Tarefa demasiado simples.
\end{itemize}

\textbf{Teste 4 - Número de cliques para seleccionar o nível 1}

\begin{itemize}

\item Resultados: 2 cliques.

\item Resultados óptimos alcançados.

\item Tarefa demasiado simples.
\end{itemize}

\textbf{Teste 5 - Tempo para matar três vírus}

\begin{itemize}

\item Resultados: 8 segundos.

\item Teste passado com sucesso.
\end{itemize}

\textbf{Teste 6 - Tempo para mover a nave de um canto do ecrã para o oposto}

\begin{itemize}

\item Resultados: 3 segundos.

\item Teste passado com sucesso.
\end{itemize}

\textbf{Teste 7 - Número de balas disparadas para matar um vírus}

\begin{itemize}

\item Resultados: 3 balas.

\item Teste passado com sucesso.
\end{itemize}

Com estes resultados e com algum \textit{feedback} que obtivemos posteriormente, podemos concluir que é necessário um botão para o menu inicial, o menu de informação dever ser remodelado e alterar a sequência Jogar $\rightarrow$ Escolher nível para Escolher nível $\rightarrow$ Jogar.

\section{Protótipo Funcional}

Nestes testes pretendemos avaliar a dificuldade do nível e a rapidez do movimento dos inimigos. Visto que no protótipo anterior explorámos mais a questão dos menus, neste focámo-nos maioritariamente na jogabilidade.\\

\textbf{Teste 8 - Número de balas disparadas para matar um vírus}

\begin{itemize}

\item Resultados: 4 balas.

\item Teste passado com sucesso.
\end{itemize}

\textbf{Teste 9 - Número de vidas perdidas até ao fim do nível 1}

\begin{itemize}

\item Resultados: 0 vidas.

\item Resultados óptimos alcançados.
\end{itemize}

\textbf{Teste 10 - Pontuação do nível 1}

\begin{itemize}

\item Resultados: 240 pontos.

\item Resultados óptimos alcançados.
\end{itemize}

\textbf{Teste 11 - Tempo para terminar o nível 1}

\begin{itemize}

\item Resultados: 26 segundos.

\item Resultados esperados alcançados.
\end{itemize}

Relativamente a estes resultados verificámos que os utilizadores ambientaram-se bem ao jogo, visto que os resultados óptimos e esperados foram sempre alcançados. Nestes testes não recebemos qualquer opinião de uma possível alteração.

\section{Protótipo Final}
No protótipo final o grande objectivo era tentar perceber se o tutorial estava minimamente intuitivo para os utilizadores conseguirem concluir o \textit{demo}.

\textbf{Teste 12 - Tempo para aceder à informação sobre \textit{spywares}}

\begin{itemize}

\item Resultados: 9 segundos.

\item Resultados mínimos alcançados.
\end{itemize}

\textbf{Teste 13 - Tempo para escolher a arma 2}

\begin{itemize}

\item Resultados: 1 segundo.

\item Resultados óptimos alcançados.
\end{itemize}

\textbf{Teste 14 - Número de tentativas para terminar o demo}

\begin{itemize}

\item Resultados: 2 tentativas.

\item Resultados esperados alcançados.
\end{itemize}

\textbf{Teste 15 - Tempo para derrotar o \textit{boss} do demo}

\begin{itemize}

\item Resultados: 28 segundos.

\item Resultados mínimos alcançados.
\end{itemize}

Nos resultados deste protótipo os resultados mínimos foram alcançados, à excepção de um utilizador que não conseguiu cumprir os dois últimos testes, porque a metáfora de perigo$\rightarrow$protecção talvez não tenha sido entendida e provavelmente terá que sofrer alterações. Quanto ao \textit{feedback} dos utilizadores, devemos evitar o uso do rato e quando surgir um ecrã de ajuda fazer reset à posição da nave.