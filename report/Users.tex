\chapter{Utilizadores e Nós}
\label{chap:users} 
%The description of all contacts with users and the results and consequences from such contacts. This includes:
%
%    Interviews and surveys
%    Cultural probes
%    Workshops
%    Usability tests
%    Any other contacts with the users
Entervistas e questionários
No primeiro contacto com o grupo de foco fizemos um pequeno questionário para perceber o quanto conhecem a Cloud e, após uma breve explicação, com que regularidade costumam usar e os principais usos. Também tentámos ver como é que estes gostariam de ver um jogo incorporado numa aula.\newline
\newline
\newline
Cultural Probes\newline
Usamos Cultural Probes para conhecer melhor alguns hábitos do nosso grupo de foco. Para isso decidimos dar a cada um saco com várias actividades: \newline
- Um calendário de um mês para ser preenchidos com autocolantes para cada tipo de actividade por dia na internet (social, jogos, ficheiros, etc.);\newline
- Um diário para ir escrevendo situações interessantes que vão ocorrendo durante o tempo na internet;\newline
- Postais com perguntas sobre gostos e opiniões sobre a internet;\newline
Para além destas criamos um grupo no Facebook onde contactavamos com o grupo de foco e partilhávamos posts engraçados e experiências na internet.\newline
\newline
\newline
Workshops\newline
Para o workshop com o grupo de foco decidimos realizar o Six Thinking Hats sobre como o jogo seria, e o que era necessário para o jogo ser divertido.\newline
O workshop foi realizado com 3 estudantes e não correu como esperado. Os alunos em vez de achar os chapéus interessantes, reagiram com receio de embaraço. A adoção de uma conversa aberta sobre o tema permitiu a recolha da informação pretendida.\newline
Com este chegamos à conclusão que um jogo apenas de perguntas e respostas seria bastante aborrecido, e que um jogo de aventura seria mais apelativo, como por exemplo um jogo com história e pistas. Reparamos também que os utilizadores gostam bastante de jogos de competição e comparar resultados.\newline
Para além disto notamos que nem todos os alunos têm um smartphone nem tablet, mas que a maior parte possui um portátil, mesmo que não seja topo de gama.
A maior parte dos alunos deram a entender que o jogo deve ser simples mas ter conteúdo relevante e actividades não monótonas. Também concluímos a aplicação de um videojogo num sala de aula no secundário é uma má escolha, pois será usado poucas vezes e seria monótono e pouco divertido, para além de acarretar todo o custo que a preparação de uma aula implica.\newline