\chapter{Protótipos}
\label{chap:proto}
%A description of all (functional and non-functional) prototypes developed together with an explanation of their development path (e.g., a discussion of any alternatives that were considered).

É de notar que a concepção do projecto começou com ideias baseadas na \textit{Cloud} e nos perigos subjacentes ao seu uso,
ideias essas que, no decorrer no projecto, foram mudando e tomando outras formas,
como pode ser visto pelo facto de que o objectivo do projecto descrito na Introdução (Capítulo \ref{chap:intro}) já não mencionar a \textit{Cloud}.
Apesar disso muitos protótipos desenvolvidos ainda tinham como base a ideia original.

\section{Prótotipos Não-Funcionais}
A construção dos protótipos começou após uma reunião de grupo, onde se acordou que cada elemento criaria 2 (dois) protótipos não-funcionais para um jogo que eles achassem que passasse bem a mensagem, estes foram os resultados.
\subsection{Protótipos do Ricardo}

\begin{figure}[h]
\centering
\includegraphics 
	[width = 0.75\textwidth] {Ricardo/ricardo_prototipo_1}
\caption{\label{fig:ric:prot1}} Protótipos do Ricardo
\end{figure}

Estes protótipos visavam jogos bastante diferentes. No lado esquerdo da Figura \ref{fig:ric:prot1} vemos um jogo ao estilo \textit{Hack and Slash}, o ideia por detrás deste jogo é que o jogador seria uma \textit{Ciber-Guardião} que defenderia a sua rede local de ataques de \textit{Malwares}, e a cada passo da história ir-se-ia descobrindo novos \textit{Malwares} e como nos proteger-mos deles.

Por sua vez, no lado direito da Figura \ref{fig:ric:prot1}, vemos um jogo de investigação, em que uma nova navegante da Internet teria de embarcar numa aventura para descobrir a entidade de um famoso \textit{Hacker}, que estava a destruir servidores e a espalhar medo pela Internet fora, e parar o seu reino de terror.
Ao longo do jogo, o jogador ia descobrindo os perigos dos \textit{Malwares} e como se proteger.

\subsection{Protótipos do Tiago}
\begin{figure}[h]
\centering
\includegraphics 
	[width = 0.75\textwidth] {Tiago/tiago_prototipo_1}
\caption{\label{fig:tiago:prot1}} Primeiro Protótipo do Tiago
\end{figure}

Ambos os protótipos focam-se mais na questão dos \textit{Malwares} e como se proteger deles, no caso do primeiro podemos verificar na Figura \ref{fig:tiago:prot1}, seria um jogo do género \textit{Platformer} em que o jogador controla um rapaz cujas informações foram roubadas do seu computador por um \textit{hacker} e o objectivo é recuperar os dados perdidos e derrotar os inimigos que vão surgindo, utilizando uma arma (\textit{antivírus}). No decorrer do jogo os inimigos iam alterando, em que no início poderiam ser \textit{vírus} e no nível seguinte \textit{worms}, o que implicaria também uma alteração da arma, por exemplo uma \textit{firewall}. Após alguns níveis o jogador iria por fim lutar contra o \textit{hacker}.

\begin{figure}[h]
\centering
\includegraphics 
	[width = 0.75\textwidth] {Tiago/tiago_prototipo_2}
\caption{\label{fig:tiago:prot2}} Segundo Protótipo do Tiago
\end{figure}

No segundo protótipo (Figura \ref{fig:tiago:prot2}) seria um jogo do género \textit{Tower Defense}, que consistiria em defender um disco rígido a meio do ecrã do jogo, dos ataques de \textit{vírus}. Para tal teríamos algumas defesas, uma \textit{firewall} e alguns soldados com escudos, como referência aos \textit{antivírus}. O objectivo seria evoluir as defesas, começando com o \textit{antivírus AVG} (nível um) e à medida que os pontos aumentassem o jogador poderia desbloquear o segundo nível e aumentar a defesa de um dos flancos. À medida que o jogo avançasse os \textit{vírus} iam aumentando de tamanho e consequentemente tornavam-se mais fortes, o que seria necessário que as defesas daquele flanco estivessem suficientemente fortes para o derrotar. Este protótipo, ao contrário do anterior, não consiste em derrotar \textit{bosses} mas sim sobreviver a \textit{waves} de ataques. E como modo \textit{multiplayer} teríamos dois jogadores a jogarem ao mesmo tempo e quem sobrevivesse mais tempo ganhava.


\subsection{Protótipos do Ian}

O primeiro protótipo (Figura \ref{fig:ian:prot1}) foi um rail-shooter, onde se controlava uma nave, que transportava uma mensagem importante, e disparava caracteres de password. O objectivo seria levar a mensagem de um ponto a outro, protegendo-a de vários \textit{Vírus} e \textit{bosses} que iriam aparecendo ao longo dos níveis. Seria também possível apanhar \textit{power ups}, como escudos de Antivírus. Haveria dois modos, um modo campanha que seria uma série de níveis, e um modo infinito, que seria um nível que apenas acabaria quando o jogador ficasse sem vidas. Este modo infinito permitiria guardar a pontuação numa tabela para comparar com a de outros jogadores.

\begin{figure}[h]
\centering
\includegraphics 
	[width = 0.75\textwidth] {Ian/ian_prototipo_1}
\caption{\label{fig:ian:prot1}} Primeiro Protótipo do Ian
\end{figure}

O segundo protótipo (Figura \ref{fig:ian:prot2}) era um \textit{God Game}, onde o jogador controlava um Deus das Nuvens (\textit{Clouds}). Esse Deus seria venerado inicialmente por uma vila, e o objectivo seria expandir essa vila, ganhando assim pontos. Com estes pontos seria possível desbloquear poderes, como por exemplo \textit{Firewalls} para proteger a vila de inimigos. Os inimigos poderiam ser vários tipo de vírus (\textit{Worms} seriam representados por serpentes gigantes, \textit{Trojans} cavalos, etc), ou outras vilas hostis, representando \textit{Hackers}. A movimentação dos habitantes pelas vilas seriam assim uma metáfora para a troca do informação pela \textit{Cloud}, e a expansão da vila o crescimento da \textit{Cloud}.

\begin{figure}[h]
\centering
\includegraphics 
	[width = \textwidth] {Ian/ian_prototipo_2}
\caption{\label{fig:ian:prot2}} Segundo Protótipo do Ian
\end{figure}

\subsection{A Escolha}

Depois de cada elemento ter revelado os seus protótipos, foi feita uma escolha de qual deles seria usado para futuros passos, como a criação de um protótipo funcional.

Os protótipos que mais chamaram à atenção, pela sua criatividade e possíveis dinâmicas de jogo, foram o \textit{God Game}, que oferecia uma metáfora forte sobre a \textit{Cloud} e \textit{Malwares}, e o \textit{Hack and Slash}, que mostrava um jogo maior e mais complexo que implicava uma evolução da personagem e uma história forte. Infelizmente, por falta de conhecimentos para criar projectos assim tão grandes, decidiu-se optar por soluções mais pequenas.

Assim, as possíveis escolhas foram o \textit{Railshooter} e o \textit{Platformer}. Ambos ofereciam uma maneira relativamente fácil de passar a mensagem, sem ser necessário um elevado nível de conhecimento num plataforma para fazer a sua implementação.

A escolha final foi o Railshooter, oferecendo ao mesmo tem uma jogabilidade simples e uma metáfora que passava lindamente a mensagem.

\emph{O transporte de uma mensagem electrónica que se tem de defender de ataques de Malwares.}

\section{Protótipos de Baixa Fidelidade}

Depois da escolha do protótipo não-funcional estar feita, é altura de criar os protótipos de baixa fidelidade.

Para a criação destes protótipos servimo-nos de duas ferramentas.
\begin{itemize}
\item FluidUI
\begin{itemize}
\item Ferramenta para criação de protótipos de aplicações mobile.
\item \url{https://www.fluidui.com/}
\end{itemize}
\item Phaser
\begin{itemize}
\item Motor de Jogo em HTML5 para jogos em Desktop e Mobile.
\item \url{http://phaser.io/}
\end{itemize}
\end{itemize}

Usamos o FluidUI para a criação de Menus que nos ajudariam a navegar pelos menus de jogo (Figura \ref{fig:fluidui})\footnote{Ver figuras nos anexos}, tendo tendo sido modificado mais tarde para corresponder a sugestões oferecidas.

Pode ver o protótipo de menus aqui: \url{https://www.fluidui.com/editor/live/preview/p_yseSDiTk0xkn4E00n0aHrnrKT242USLL.1416862868340}.

Quanto ao jogo em si, tomámos como base um jogo oferecido no website do Phaser, o Space Invader ( visite \url{http://examples.phaser.io/_site/view_full.html?d=games&f=invaders.js&t=invaders} para jogar). Partindo desta base criamos o nosso primeiro protótipo (Figura \ref{fig:phaser}).

\section{Protótipo Final}

Depois de algumas iterações com apresentações (Figura \ref{fig:pres}), entrevistas e testes de usabilidade que levaram a uma constante modificação do protótipo, criou-se o protótipo final.
Este protótipo é um Demo do jogo que imaginámos construir, permitindo ao jogador ver os mecanismos de jogo e ao mesmo tempo passar a mensagem séria (que é o objectivo do projecto).

Como é possível ver nas seguintes Figuras.

\begin{figure}[h]
\centering
\includegraphics 
	[scale = 0.5] {SpywareInfo}
\caption{\label{fig:spyw}} Menu de Informação Sobre o Spyware
\end{figure}

\begin{figure}[h]
\centering
\includegraphics 
	[scale = 0.5] {GameplayExample}
\caption{\label{fig:gameplay}} Imagem do Jogo a ser jogado
\end{figure}